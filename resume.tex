\documentclass[11pt]{moderncv}
\usepackage[utf8]{inputenc}
\usepackage[russian]{babel}
\usepackage{xcolor}

\moderncvstyle{casual}       
\moderncvcolor{purple}   
\firstname{Кирилл}
\familyname{Куницын}
\email{kirill.kunitsyn@gmail.com}
\mobile{+79841048459}
\social[github]{BerserkTears}
\homepage{www.stockgolm.xyz}  
\address{Санкт-Петербург}

\begin{document}
\makecvtitle
\section*{Образование}
\cventry{2021-now}{Университет ИТМО}{}{}{}{
    Направление подготовки (специальность):Информационные системы и технологии\\
    Образовательная программа: Программирование и интернет-технологии
    }
\section*{Опыт}

    \cventry{...--2021}{\colorbox{lightgray}{Школа}} {}{}{}{
        В процессе обучения в школе я выступал на многих международных выставках
        и конференциях со своими проектами. Освоил такие технологии как \texttt{Arduino},
        \texttt{RaspberryPi}, работа с \texttt{API} в \texttt{JavaScript}.
    }

    \cventry{2019}{\colorbox{lightgray}{Хакатон} Моя Проффесия - IT 2019} {}{}{}{
        В процессе хакатона я выступал в роли тимлидера и освоил 
        базовые навыки backend и frontend разработки, а также получил опыт работы 
        в команде.
    }

    \cventry{2020}{\colorbox{lightgray}{Хакатон} Моя Проффесия - IT 2020} {}{}{}{
        В этот раз я выступил в роли программиста и разрабатывал вебсайт, в основе
        которого были \texttt{PHP} и \texttt{MySQL}, также я работал с фреймворками
        \texttt{RedBeanPHP} и \texttt{Bootstrap}.
    }

\section*{Навыки}
    \cvitemwithcomment{C/C++}{Знаю синтаксис и пишу программы}{Изучается в университе}
    \cvitemwithcomment{PHP}{Знаю синтаксис и пишу программы}{Примеры работ на гитхабе}
    \cvitem{MySQL}{Хорошо понимаю устройство баз данных}
    \cvitem{Алгоритмы}{Знаю основные алгоритмы и структуры данных}
    \cvitem{Also}{CMake, HTML, GitHub, знаю принципы Agile и Waterfall разработки, LaTEX }
\newpage
\section*{О себе}
    \cvitem{}{Я стремлюсь получать новые знания. Я выиграл Президентский грант за высокие достижения в области математики и информатики, являюсь лауреатом именной стипендии Первого Президента РС(Я) М.Е. Николаева «Знанием победишь!», являлся действительным членом Малой Академии Наук Республики САХА (Якутия).}
\end{document}